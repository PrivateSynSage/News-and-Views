\documentclass[twocolumn]{article}
\usepackage[utf8]{inputenc}
\usepackage{blindtext}
\usepackage{microtype}
\usepackage{multicol}
\usepackage{natbib}
\usepackage[]{graphicx}

\setlength{\parskip}{1em}
\setlength{\parindent}{0em}

\title{{\Large Methylation Can Lead to Differential Transgene Expression}}
\author{\large Ryan Young}
\date{August 1987 - with one anachronistic source}

\begin{document}

\maketitle
% \clearpage

% \twocolum

As of yet, no one understands why parthenogenesis fails in mammals who receive chromosomes from both parents \citep{solter}\citep{surani}. It has been hypothesized there are extra-genetic\footnote{Beyond standard nucleotide sequences}, molecular differences between the male and female inherited material, and that these are ``imprinted'' before birth \citep{surani}. Furthermore, only candidate mechanisms have been suggested but have not been connected to an expression consequence \citep{reik}. Swain, Stewart, and Leder in their recent article have produced an example of such a mechanism, leading to an expression consequence.

While studying the effect of an oncogene, c-myc, they came across a peculiar animal. This particular rodent expressed their gene containing c-myc, a transcription factor important (potentially) to cycle progression\citep{cymc}, in heart tissue, but \textit{only in males}, nowhere else. They set out to further characterize this oddity, with an eye to the phenomena of methylation. Previous research had shown genes methylated when inherited from the mother but not the father \citep{surani}, without seeing an expression consequence.

Using nearly 700 descendants of the mouse, they assayed for expression. They sampled RNA
from a host of organs, and mixed them with a complementary cDNA probe spanning 365bp segment between two exons. After digestion with RNAse, protected by hybridization between strands, only RNA with complementary sequences remained. Among the remnants were fragments of RNA for the two c-myc exons, and for the processed transgene. The latter only showed in the heart. They then crossed animals who either carried or did not carry this transgene, and surprisingly, noted among offspring carrying the gene: if the father carried, but not the mother, there was 100\% expression, and vice-versa, if the mother, but not the father, carried, there was 0\% expression, both across many animals. 

To pull the lid back further, they also examined DNA substrata giving rise to these RNA. Previous work had implied methylation might be the culprit. Stretching back a few years prior, it had been noticed that sperm pronuclear material modified methylation\citep{} and that methylation seemed to have an effect on transcription\citep{}. Hence, Swain, Stewart, and Leder chose to apply an enzyme that could recognize/cut transgene sequences; not, however, if they were methylated. Looking at the fragments on a Southern blot, the genetic material restricted into smaller fragments as described above---exclusively for transgenes passed from crosses of a male carrier with non-carrier females, not vice-versa. Additionally, the undermethylation pattern reared, again, only in the heart. This squared the circle, demonstrating 100\% correlation between the transgene's methylation status and expression.

But many questions remain unanswered. 

% Critically, this may confirm our suspicions that methylation plays a role in the process.


 

% \end{multicols}



\end{document}
